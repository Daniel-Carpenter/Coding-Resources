\documentclass[handout,t]{beamer}
\usepackage[portuges]{babel}
\usepackage[utf8]{inputenc}
\usepackage[alf]{abntex2cite}
\usepackage{graphicx}
\usepackage{float} 
\usepackage{geometry}
\usepackage{amsmath}
\usepackage{amssymb}
\usepackage{url}
\usepackage{pgfpages}
\usepackage{enumerate}
\usepackage{color}
\usepackage{ifthen}
\usepackage{capt-of}
\usepackage[authoryear]{natbib}
\usetheme{Berlin}   
%\usecolortheme{ufrn}
\setbeamertemplate{caption}[numbered]

% COLORS ===================================================================================================
\usepackage{color}
%R G B
\DefineNamedColor{named}{azul2}{RGB}{40,96,127}
\DefineNamedColor{named}{azul1}{RGB}{4,151,229}
\DefineNamedColor{named}{azul3}{RGB}{1,41,64}
\DefineNamedColor{named}{laranja1}{RGB}{255,146,5}

\mode<presentation>
\setbeamercolor{author in head/foot}{bg=azul1, fg=white}
\setbeamercolor{title in head/foot}{bg=azul2, fg=white}
%\setbeamercolor{institute in head/foot}{fg=instituteColor}
\setbeamercolor{date in head/foot}{bg=azul3, fg=white}

\setbeamercolor{alerted text}{fg=azul1!80!laranja1}
\setbeamercolor*{palette primary}{fg=white,bg=azul2}
\setbeamercolor*{palette secondary}{fg=laranja1,bg=white}
\setbeamercolor*{palette tertiary}{fg=white,bg=azul3}
\setbeamercolor*{palette quaternary}{fg=white,bg=laranja1}
\setbeamercolor*{structure}{fg=azul3,bg=white}
\setbeamercolor{frametitle}{fg=white,bg=azul3}
\mode<all>

\defbeamertemplate*{footline}{themeName}
{
	\leavevmode
	\hbox{
		\begin{beamercolorbox}[wd=.333333\paperwidth,ht=2.25ex,dp=1ex,center]{author in head/foot}
			\usebeamerfont{author in head/foot}\insertauthor{}
		\end{beamercolorbox}
		
		\begin{beamercolorbox}[wd=.333333\paperwidth,ht=2.25ex,dp=1ex,center]{title in head/foot}
			\usebeamerfont{title in head/foot}\text{Portfolio Optimization Program}
		\end{beamercolorbox}
		
		\begin{beamercolorbox}[wd=.333333\paperwidth,ht=2.25ex,dp=1ex,right]{date in head/foot}
			\usebeamerfont{date in head/foot}\insertshortdate{}\hspace*{2em}\insertframenumber{} / \inserttotalframenumber\hspace*{2ex}
		\end{beamercolorbox}}
	\vskip0pt
}

% TITLE PAGE ===============================================================================================
    \title[Title]{Stock Portfolio Optimization and Automation Program}
    \date{April 2020}
    \author[Author]{Daniel Carpenter}
    \institute[University]{University of Oklahoma}
    
    \vfill
    \pagebreak{}
    \begin{spacing}{1.0}
    \bibliographystyle{jpe}
    \bibliography{FinalProject_Carpenter.bib}
    \end{spacing}
    \vfill

% SLIDES BODY ==============================================================================================
    \begin{document}
    % Outline
        \frame{\titlepage}
        \section[]{}
            \begin{frame}{Outline of Slides}
            	\tableofcontents
            \end{frame}
    
    % Introduction -----------------------------------------------------------------------------------------
    \section{Introduction}
        \begin{frame}{Introduction}
        	\begin{doublespacing}
            	\begin{enumerate}
            	    \item Basics of a stock portfolio 
            	    \item What it means to optimize a portfolio 
            	    \item Modern Portfolio Theory and Capital Market Line 
            	    \item Spoiler alert: this method isn't mainstream anymore, but it can be very useful
            	\end{enumerate}
            \end{doublespacing}
        \end{frame}
    
    % Data  ------------------------------------------------------------------------------------------------
    \section{Data}
        \begin{frame}{Using Yahoo Finance's API}
        	\begin{doublespacing}
            	\begin{enumerate}
            	    \item Stock Data from Yahoo Finance's API
            	    \item Package allows infinite number of stock inputs
            	    \item Package easily integrates S\&P 500, allowing for broad sample
            	\end{enumerate}
            \end{doublespacing}
        \end{frame}
        
        \begin{frame}[allowframebreaks]{Manipulating the Dataset}
            \begin{doublespacing}
            	\begin{enumerate}
            	    \item Data cleaning mainly includes formatting issues
            	    \item Added columns for excess return calculation
            	    \item Table 1: Example of the API Export 
                            \begin{center}
                                % latex table generated in R 3.6.0 by xtable 1.8-4 package
                                % Mon Apr 13 09:21:39 2020
                                \begin{table}[ht]
                                \centering
                                \begin{tabular}{rllrr}
                                  \hline
                                 & date & stockName & stockPrice & stockReturn \\ 
                                  \hline

                                  1 & 2020-02-03 & MSFT & 173.90 & 0.07 \\ 
                                  2 & 2020-03-02 & MSFT & 172.79 & -0.01 \\ 
                                  3 & 2020-04-01 & MSFT & 152.11 & -0.12 \\ 
                                   \hline
                                \end{tabular}
                                \end{table}
                            \end{center}
                    \framebreak
                    \item Drops null values for firms who were not publicly traded during date range
                    \item Pivots data so that excess return on stocks are column variables 
            	\end{enumerate}
            \end{doublespacing}
        \end{frame}
     
    % Methods ----------------------------------------------------------------------------------------------
    \section{Methods}
        \begin{frame}[allowframebreaks]{Methods}
        	\begin{doublespacing}
            	\begin{enumerate}
            	    \item Calculating Excess Returns
            	        \begin{equation} \label{Excess Returns}
                            Returns_{Excess} = Return_{Stock_{i}} - RiskFreeRate
                        \end{equation}
            	    \item Shows abnormal returns beyond United States Treasury-Bill 
            	    \item Why the T-Bill? T-Bill's are assumed "riskless" assets because they experience low fluctuations in value and are highly liquid
            	    
            	    \framebreak
            	    \item Finding Variance and Covariance between Stocks allows for better optimization
                         \begin{equation} \label{Variance-Covariance-Matrix}
                                Variance-Covariance-Matrix = 
                                \frac{X^{T}_{Excess Returns} * X_{Excess Returns}}
                                {Number Of Stocks_{X_{Excess Returns}}} 
                        \end{equation}
            	    
            	    \framebreak
            	    \item Linear Optimization with rGLPK package
            	    \item Program outputs optimal weights in each stock
                        % latex table generated in R 3.6.0 by xtable 1.8-4 package
                        % Mon Apr 27 12:21:53 2020
                        \begin{table}[ht]
                        \centering
                        \begin{tabular}{rlrr}
                          \hline
                         & Stock Name & Stock Weight & Dollar Investment \\ 
                          \hline
                        1 & ADS & 0.51 & 5144.09 \\ 
                          2 & IR & 0.49 & 4855.91 \\ 
                           \hline
                        \end{tabular}
                        \end{table}
                    \item Note that the program is set to a 10,000 dollar investment, and the "Stock Weight" should be read as percentages.
            	    \framebreak
            	    \item Program outputs standard deviation (risk), expected return, and Sharpe Ratio of the optimized Portfolio
                        % latex table generated in R 3.6.0 by xtable 1.8-4 package
                        % Mon Apr 27 12:21:53 2020
                        \begin{table}[ht]
                        \centering
                        \begin{tabular}{rr}
                          \hline
                         & Value \\ 
                          \hline
                        Risk & 0.29 \\ 
                          Expected Return & 0.15 \\ 
                          Sharpe Ratio & 0.45 \\ 
                           \hline
                        \end{tabular}
                        \end{table}
                    \item Interpretation of Risk: Given the contents of the portfolio, the expected variation in stock prices equal 29 percent
            	\end{enumerate}
            \end{doublespacing}
        \end{frame}
        
    % Findings ---------------------------------------------------------------------------------------------
    \section{Findings}
        \begin{frame}{Findings}
        	\begin{doublespacing}
            	\begin{enumerate}
            	    \item Many alternatives to passive investments
            	    \item Not ideal to use this tool for active investments. 
            	    \item According to the Wall Street Journal, you will be beat 90 percent of time by a passive fund.
            	    \item Wall Street's computer algorithms trade in fractions of a second, leaving the individual investor in the dust
            	\end{enumerate}
            \end{doublespacing}
        \end{frame}
        
    % Importance of Topic ----------------------------------------------------------------------------------
    \section{Importance of Topic}
        \begin{frame}{Importance of Topic}
        	\begin{doublespacing}
            	\begin{enumerate}
                    \item This tool can be used for investment decision making
                    \item Easily bridges uncertain investors to information, which can be powerful
            	\end{enumerate}
            \end{doublespacing}
        \end{frame}
    
    
\end{document}